% !TeX root = ../thuthesis-example.tex

% 中英文摘要和关键字

\begin{abstract}
  人工智能的高速发展为各行各业带来了革命性的转变,也逐渐在全球范围内成为引领性技术。
  世界各国和主要经济体纷纷加快人工智能战略布局,并相应出台了一系列人工智能发展规划、政策和法律法规。
  人工智能技术的研究和应用进入了一个新的发展阶段。

  作为全球人工智能战略的重要基石,机器学习技术已深入各行各业,但成本高、管理乱、落地难的困境却一直制约着它的发展。
  企业使用机器学习技术的投入和收益往往不成比例。
  其中,由烟囱式和手工作坊式的模型研发模式所带来的研发效率低下是造成这一困境的主要原因:
  一方面,模型研发的工作流程缺乏顶层设计,分工不明确,难以形成有效的团队协作;
  另一方面,模型研发过程中产生的资产缺乏统一的存储和管理,资产易流失,难以形成有效的知识沉淀和复用。
  
  本文聚焦于机器学习研发管理系统的研究,旨在通过构建机器学习平台,运用云原生背景下的技术和工程化的手段支撑机器学习研发的全流程,提升研发效率能,降低研发成本和管理复杂度。
  从设计的角度,本文研究了云原生机器学习平台的体系架构,为平台设计和实现提供指导性参考;
  从技术的角度,本文研究了云原生系统和机器学习两方面的关键技术,为平台的能力和功能提供支撑;
  从实现的角度,本文研究了Anylearn大数据机器学习研发管理系统的设计实现和应用情况,展示了平台在提升研发协作、资源利用率和实验复现性方面的显著成效。

  % 关键词用“英文逗号”分隔,输出时会自动处理为正确的分隔符
  \thusetup{
    keywords = {机器学习, 云原生, MLOps, 软件工程},
  }
\end{abstract}

\begin{abstract*}
  The rapid development of artificial intelligence (AI) has introduced revolutionary progress to various industries and has gradually become a global leading technology.
  The major entities all over the world are accelerating their strategic deployment of AI and have come up with tactical AI development plans, policies, and regulations.
  The research and development (R\&D) of AI technologies have entered a new stage of development.
  
  Machine learning (ML), as a foundational technology of global AI strategies, has been widely adopted.
  However, its development has been consistently constrained by challenges such as high costs, disorganized management, and difficulties in maintenance.
  The added value for enterprises adopting machine learning in their business often falls short of expectations.
  One of the reasons lies in the inefficiency of the ad-hoc fashion in ML R\&D.
  On one hand, the lack of high-level design and chaotic task assignment in the model development process makes effective team collaboration nearly impossible.
  On the other hand, the absence of centralized management for assets generated during R\&D leads to a huge complexity in reuse of knowledge.
  
  This study focuses on the research of machine learning development management systems, aiming to support the full lifecycle of machine learning development by building a cloud-native machine learning platform that leverages engineering methods and technologies.
  This approach seeks to improve development efficiency, to reduce costs, and to simplify the management.
  From the perspective of system design, this research explores the architecture of cloud-native machine learning platforms, providing a reference guidance for designing real system and its implementation.
  Then technically, this paper investigates key technologies in both cloud-native systems and machine learning, supporting the machine learning platform's capabilities and functionalities.
  Finally, the big data machine learning R\&D management system Anylearn is examined in terms of its design, implementation, and application in real-world scenarios, demonstrating its effectiveness in enhancing development collaboration, resource utilization, and experimental reproducibility.

  % Use comma as separator when inputting
  \thusetup{
    keywords* = {machine learning, cloud native, MLOps, software engineering},
  }
\end{abstract*}
