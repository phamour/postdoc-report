% !TeX root = ../thuthesis-example.tex

\chapter{总结与展望}

本文系统研究了云原生机器学习平台的体系架构、关键技术以及真实系统的设计实现。
针对机器学习研发中存在的流程缺乏顶层设计、资产缺乏统一管理、协作缺乏基础手段等挑战,本文提出了一个从能力、功能、数据、技术到交互的完整体系框架。
围绕平台的核心功能,本文研究了云原生系统和机器学习两方面的关键技术,包括容器化任务执行、资源池化调度、分布式存储管理、分布式训练、超参数调优、模型微调、模型更新等方面。
在此基础上,本文描述了真实的大数据机器学习研发管理系统Anylearn的设计实现和线上环境的应用情况,展示了平台在提升研发协作、资源利用率和实验复现性方面的显著成效,并证明其在真实的模型研发场景中的适用性和有效性。

未来的研究工作将围绕以下几个方向展开。

(1)国产化适配。
在全球人工智能战略布局的博弈中,国外软硬件生态对我国人工智能技术发展的封锁是当前人工智能领域最重大的风险点之一,卡脖子现象频发,国产化自主可控的人工智能软硬件生态则是破局的关键。
当前的云原生机器学习平台中的技术栈大多基于开源生态和国外算力基础设施,集成化的平台对于国产化软硬件的适配性有待充分验证。
Anylearn的国产化适配也是未来的重要工作之一。

(2)模型安全。
随着人工智能技术的广泛应用,模型安全问题日益凸显,如对抗攻击、隐私泄露、模型解释性等问题。
未来的研究也需要重点围绕模型安全展开,包括对抗攻击下的模型鲁棒性、隐私保护下的模型训练、模型解释性等方面。

(3)大模型底座。
自2022年底,大语言模型经历了爆炸式的增长,同时也带动了广泛意义上大模型技术的发展,如气象大模型、时序大模型等。
随着规模化效应的加剧,机器学习研发在大模型时代面临的挑战将更加复杂。
如何通过机器学习平台来支撑大模型的研发、形成大模型底座,将是未来长期的研究重点。
